\section{Executive Summary}
	Wind tunnels are used to study the aerodynamic characteristics of objects under the effect of air in motion. Characteristics  studied include a number of physical quantities, such as drag, lift, and side. These are forces exerted by the air in three dimensional space, one for each component. For laminar flow, the tunnel measures these by pulling air into the tunnel and having  devices that measure these forces on the object. At the University of Puerto Rico at Mayaguez, Department of Civil Engineering, the tunnel uses a balance that mechanically measures of these forces in a manual process. At the Microprocessor Interfacing course, the team had the opportunity to develop an electronic alternative that automates this process for two forces: drag and lift. A small wind tunnel prototype was built for the course. The objective proposed is to move from the small prototype to the complete implementation of the system on the Civil Engineering Department's wind tunnel by adding new features and improving existing components.
	
	Both hardware and software have to be improved to ensure the quality of the project, that is, the robustness of the software, the accuracy and precision of measurements, the tolerance to failure of the system and the available resolution of data. On the hardware side the project must control the drive that limits the pull of air of a room sized motor, add the feedback control algorithm that adjusts the speed in the tunnel, condition the signal from the components that measure the forces, and measure the third component, i.e., the side. On the software side, the data is not being stored for statistical or other types of analysis about the experiments. Models of the experiment, such as miniature cars, airplane components, and others can have this data along with their design and therefore can be shared with the community for collaborative study. Hence an application that shows the progress of the experiment and automatically shows the history and all data will be implemented to achieve these goals. 

	The project will be developed in 13 weeks in the facilities of the university and the end goal is to have a fully developed and deployed system. It will be used mainly in courses in Civil Engineering and Mechanical Engineering programs. Funds for the system hardware are provided by the client, Dr. Raul Zapata, professor of the Civil Engineering Department. The end date of the project is May 6th.